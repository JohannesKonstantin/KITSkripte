\documentclass{scrartcl}
\usepackage{amsmath}
\usepackage{amsfonts}
\usepackage{amssymb}
\usepackage{enumerate}
%% Support for the target language %%%%%%%%%%%%%%%%%%%%%%%%%%%%%%%%%%%%
\usepackage[ngerman]{babel}

%% Font & encoding %%%%%%%%%%%%%%%%%%%%%%%%%%%%%%%%%%%%%%%%%%%%%%%%%%%%
\usepackage[T1]{fontenc}
\usepackage[utf8]{inputenc}

%% Custom headings %%%%%%%%%%%%%%%%%%%%%%%%%%%%%%%%%%%%%%%%%%%%%%%%%%%%
\usepackage[explicit]{titlesec}

%% Shows the use of deprecated stuff %%%%%%%%%%%%%%%%%%%%%%%%%%%%%%%%%%
\usepackage{nag}

%% Code listings %%%%%%%%%%%%%%%%%%%%%%%%%%%%%%%%%%%%%%%%%%%%%%%%%%%%%%
\usepackage{listings}

%% Advanced math %%%%%%%%%%%%%%%%%%%%%%%%%%%%%%%%%%%%%%%%%%%%%%%%%%%%%%
\usepackage{amsmath}
\usepackage{amssymb}
\usepackage{amsthm}

%% Table with extended features %%%%%%%%%%%%%%%%%%%%%%%%%%%%%%%%%%%%%%%
\usepackage{tabularx}

%% Graphic handling %%%%%%%%%%%%%%%%%%%%%%%%%%%%%%%%%%%%%%%%%%%%%%%%%%%
\usepackage{graphicx}

%% Table coloring %%%%%%%%%%%%%%%%%%%%%%%%%%%%%%%%%%%%%%%%%%%%%%%%%%%%%
\usepackage[table]{xcolor}

\usepackage{setspace}

%% Formatting %%%%%%%%%%%%%%%%%%%%%%%%%%%%%%%%%%%%%%%%%%%%%%%%%%%%%%%%%

\renewcommand{\labelitemii}{$\bullet$}

\setlength\parindent{0pt}

\newcommand{\qq}[1]{\glqq #1\grqq}

\newcommand{\q}[1]{\glq #1\grq}

\newcommand{\definition}[1]{\glq #1\grq}

\renewenvironment{proof}{{\bfseries Beweis }}{\qed}

\newenvironment{example}{{\bfseries Beispiel }}{}

\newtheorem{theorem}{Satz}

\begin{document}
\title{Analysis I Skript}
\date{9.11.2013}
\author{Rene Brandel und Rudolf Biczok}
\maketitle
\section{Grundlagen}

\subsection{Mengen}

Angaben von Mengen durch Aufzählungen

$M=\left\lbrace a,b,c \right\rbrace$ oder $M=\left\lbrace Kirche, Dorf\right\rbrace$

bekannte Mengen:
\begin{itemize}
  \item $\emptyset$ leere Menge
  \item $\mathbb{N} = \left\lbrace 1,2,3,\ldots \right\rbrace$ natürliche Zahlen
  \item $\mathbb{Z} = \left\lbrace \ldots,-3,-2,-1,0,1,2,3,\ldots \right\rbrace$ ganze Zahlen
  \item $\mathbb{Q} = \left\lbrace \frac{m}{n}|m \in \mathbb{Z},n \in \mathbb{N} \right\rbrace$ Rationale Zahlen
\end{itemize}

\textbf{Achtung:} $\left\lbrace \emptyset \right\rbrace$ hat ein Element (nämlich eine leere Menge)!

\subsubsection{Syntax}

\begin{itemize}
  \item $x \in M$ $x$ ist Element von $M$
  \item $x \notin M$ $x$ ist nicht Element von $M$
  \item $M \subset N$ $M$ ist Teilmenge von $N$
  d.h. für alle $x \in M$ ist auch $x \in N$ \\
  \textbf{Achtung:} Bei $M \subset N $ ist auch $M = N$ möglich \\
  \textbf{Immer:} $\emptyset \subset M$,in jeder Menge
  \item $M=N: M \subset N \wedge N \subset N$
  \item $M \cup N := \left\lbrace x|x \in M \wedge x \in N\right\rbrace$ \\
  \textbf{Disjunktion:} $M$ und $N$ sind disjunkt wenn $M  \cup N = \emptyset$ Schnittmenge
  \item $M \cap N := \left\lbrace x|x \in M \vee x \in N\right\rbrace$ Vereinigungsmenge
  \item $M \backslash N := \left\lbrace x|x \in M \vee x \notin N \right\rbrace$ Differenz
  \item $M \times N:= \left\lbrace (x,y)|x \in M, y \in N \right\rbrace$ Produktmenge \\
  $M_1 \times N_2 \times \ldots \times M_n := \left\lbrace \underbrace{(x_1,x_2,\ldots,x_n)}_{\text{n-Tupel}} : x_j \in M_j, j= 1,\ldots,n \right\rbrace$
\end{itemize}

\subsubsection{Satz 1: \qq{Naiver} Mengenbegriff nach Cantor}

\qq{Unter einer \q{Menge} verstehen wir jede Zusammenfassung M von bestimmten wohlunterschiedenen Objekten m unserer Anschauung oder unseres Denkens (welche die \q{Elemente} von M genannt werden) zu einem Ganzen.}

\subsubsection{Potenzmenge von M}

$2^M = \mathcal{P}(M) := \left\lbrace A | A \subset M \right\rbrace$

\textbf{immer:} $ M \in \mathcal{P}(M), \emptyset \in \mathcal{P}(M)$ 

\begin{example}{$\mathcal{P}(\emptyset) = \left\lbrace \emptyset \right\rbrace$}
\end{example}

\subsubsection{Satz 2: Funktionen}

Eine Funktion oder Abbildung $f : x \to y$ besteht aus einem Definitionsbereich $X$ und eine Abbildungsvorschrift, die jedem $x \in X$ genau ein Element $y \in Y$ zuordnet.

\textbf{Schreiben} $y=f(x)$, erfordert auf $x \mapsto f(x)$
\begin{align*}
  f : X &\to Y\\
  x &\mapsto f(x)
\end{align*}

\begin{example}
\begin{align*}
  f : \mathbb{N} &\to \mathbb{N}\\
  x &\mapsto f(x) = 2x
\end{align*}
\end{example}

\subsubsection{Satz 3: Graph}

Sei $f:x \to y$ eine Funktion

$Graph(f) = G(f) = \left\lbrace (x,f(x)) : x \in X \right\rbrace$

$G(f) \subset X \times Y$

Zwei Funktionen $f_1 : X \to Y, f_2 : X \to Y$ sind gleich, wenn $G(f_1) = G(f_2)$. D.h. falls $f_1(x)=f_2(x)$ für alle $x \in X$.

\subsubsection{Funktionsraum}

$Y^X = Abb(X,Y) = $ Menge aller Funktionen $f : X \to Y$

\subsubsection{Bild}

Wenn $A \subset X$:

$f(A):=\left\lbrace y \in Y : \text{ Es gibt ein } x \in A : y=f(x) 
\right\rbrace$

$f(A):=\left\lbrace f(x) : x \in A \right\rbrace$

Bild von $A$ (unter $f$)

\subsubsection{Urbild}

Wenn $B \subset Y$ 

$f^{-1}(B):=\left\lbrace x \in X : f(x) \in B \right\rbrace$

Urbild von $B$ (unter $f$)

\subsubsection{Eigenschaften von Funktionen}

$f(X)$ ist das Bild von $f$

$f : X \to Y$ ist:

\begin{description}
\item[injektiv:] falls aus $x_1,x_2 \in X$ und $f(x_1)=f(x_2)$, dann gilt bereits $x_1=x_2$. 

\item[surjektiv:] falls $f(X) = Y$.

\item[bijektiv:] falls surjektiv und injektiv zugleich.
\end{description}

\subsubsection{Umkehrabbildung / Umkehrfunktion}

Ist $f : X \to Y$ bijektiv, so existiert zu jedem $y \in Y$ genau ein $x \in X$ mit $y =f(x)$. Die Inverse zu $f$ ist die Funktion: 
\begin{align*}
  f^{-1} : Y &\to X\\
  y &\mapsto \text{ Urbild von } Y \text{ unter } f
\end{align*}

\begin{example}
\begin{align*}
  f : \mathbb{N} & \to     \mathbb{N}\\
                    x & \mapsto 2x
\end{align*}
$ f^{-1}(\{3\}) = \emptyset $

$\rightarrow$ ist nicht bijektiv

\[P : N \to \text{gerade natürliche Zahlen}\]
\begin{align*}
  f : P(\mathbb{N}) & \to     P(\mathbb{N})\\
                       x & \mapsto 2x
\end{align*}
\end{example}

$\rightarrow$ ist bijektiv

$f^{-1}(y)=\frac{y}{2} \in \mathbb{N},y=$ gerade natürliche Zahl.

\subsubsection{Komposition}

Sei $f : X \to Y, g : W \to Z$ mit $f(X) \subset W$

$h:= g \circ f$ ($g$ ist verknüpft mit $f$)
$h(x):= (g \circ f)(x):=g(f(x))$

\subsubsection{Identität}

\begin{align*}
  id_M : M & \to     M\\
              x & \mapsto x
\end{align*}

Es gilt: $f : M \to N$ ist bijektiv, dann gilt:

\begin{enumerate}
 \item $f^{-1} : N \to M \text{ existiert} $
 \item $f^{-1} \circ f = id_M$
 \item $f \circ f^{-1} = id_N$
\end{enumerate}

\subsubsection{Restriktion und Fortsetzung}

Seien $f : X \to Y$ und $g : X \to A$ Funktionen und $A \subset X$

\textbf{$g=f|_A$ heißt Restriktion (oder Einschränkung) von $f$ auf $A$:}

\begin{align*}
  g := f|_A : A & \to     Y\\
                   x & \mapsto f(x)
\end{align*}

\textbf{$f|_A := g$ heißt Fortsetzung von $g$ auf $X$:}

\begin{align*}
  f|_A := g : X & \to     Y\\
                   x & \mapsto g(x)
\end{align*}

\begin{example}
\begin{align*}
  g : [0,\infty) & \to     [0,\infty)\\
                    x & \mapsto x^2
\end{align*}
\begin{align*}
  f : (-\infty,\infty) & \to     [0,\infty)\\
                          x & \mapsto x^2
\end{align*}
\end{example}

\subsection{Induktion}

Sei $\mathbb{N} = \{1,2,3,\ldots\}$
$\mathbb{N}_0=\mathbb{N} \cup \{0\}$

\subsubsection{Satz 4: Prinzip der vollständigen Induktion}

Eine Teilmenge $M \subset \mathbb{N}$ erfülle:

\begin{description}
 \item[a)] (IA: Induktionsanfang) $1 \in M$.
 \item[b)] (IS: Induktionsschritt/Induktionsschritt) \\
    Falls $k \in M$ ist, demnach ist auch $k+1 \in M$
\end{description}

dann ist $M = \mathbb{N}$.

\begin{example}
Aussage: Für alle $n \in \mathbb{N}$
\[A(n) = 1+2+\ldots+n = \frac{n(n+1)}{2}\]
\[M:=\{n \in \mathbb{N} : A(n) \text{ ist wahr }\} \subset \mathbb{N}\]
Wissen: $1 \in M$, da $A(1)$ wahr ist

Annahme: 
\[k \in M \Longrightarrow A(k) \text{ ist wahr }\]
\[A(k+1):1+2+\ldots+k+(k+1)= \frac{(k+1)(k+2)}{2}\]
\[\underbrace{1+2+\ldots+k}_{\frac{k(k+1)}{2}}+(k+1)= \frac{k(k+1)}{2}+(k+1)=\frac{(k+1)(k+2)}{2}\]

$\Longrightarrow k+1 \in M$ falls $k \in M$ ist! also wegen Satz 4: $M=\mathbb{N}$!
\end{example}

\subsubsection{Satz 5: Beweis durch vollständige Induktion}

Für alle $n \in \mathbb{N}$ seien Aussagen $A(n)$ gegeben.

Ferner sei:

\begin{description}
 \item[(IA)] $A(1)$ ist wahr.
 \item[(IS)] Unter der Annahme, dass für ein $k \in \mathbb{N}$ die Aussage $A(k)$ wahr ist, ist dann auch $A(k+1)$ wahr
 \item[(IS)] Aus $A(n)$ wahr für $n=k$ folgt $A(n)$ wahr für $n=k+1$ \\
   Dann ist $A(n)$ wahr f+r alle $n \in \mathbb{N}$
\end{description}

\begin{proof}

Setze man $M:=\{n \to \mathbb{N}:A(n) \text{ wahr }\}$

$M \subset \mathbb{N}$

\begin{enumerate}
  \item Wegen (IA) $1 \in M$
  \item Wegen (IS) sei $k \in M$, also $A(k)$ wahr, also $A(k+1)$ wahr, also $k+1 \in M$ 
\end{enumerate}

Wegen Satz 4 fertig!

\end{proof}

\begin{example}{Summen und Produkte}

Seien $a_1, \ldots, a_n$ Zahlen

\textbf{Definition:} Teilsumme
\[S_k \text{ durch } S_1 := a_1\]
\[\text{für } k \in \mathbb{N}: S_{k+1}:=S_k+a_{k+1}\]
\[\text{Setze } a_1 + \ldots + a_n = \sum_{i=1}^n a_j:=S_n \]

$\rightarrow$ Beispiel für eine rekursive Definition

\textbf{Definition:} Produkte
\[ p_1:=a_1\]
\[ p_{k+1}:=p_k*a_{k+1} \]
\[ a_1*...*a_n=\prod_{j=1}^n a_j:=p_n\]
\[ a^n = \underbrace{a*...*a}_{\text{n-mal}}:=\prod_{j=1}^n a\]

\textbf{Setzen:}
\[ \sum_{j=1}^0 a_j := 0 \quad \quad \quad \prod_{j=1}^0 a_j := 1 \quad \quad \quad a^0 = 1\]

\end{example}

\begin{example}{Geometrische Summe}
\[\text{Sei } a \neq 1, n \in \mathbb{N}_0\]
\[\Longrightarrow \sum_{j=0}^{n} a^j = \frac{a^{n+1}-1}{a-1}\]

\begin{proof}{1: Induktion}

(IA) hier $n=0$
\[\sum_{j=0}^0 a^0 = 1 = \frac{a^1-1}{a-1}\]
(IS) Wir nehmen an, dass für $k \in \mathbb{N}$ die Formel für $n=k$ wahr ist.
\[ \sum_{j=0}^k a^j = \frac{a^{k+1}-1}{a-1} \]
\[ \text{IS auf n=k+1} \]
\[ \sum_{j=0}^{k+1} a^j = \sum_{j=0}^{k} a^j +a^{k+1} \]
\[ \text{Induktionsannahme} \]
\[ = \frac{a^{k+1}-1}{a-1}+a^{k+1}=\frac{a^{k+1}-1+(a-1)a^{k+1}}{a-1} \]
\[ = \frac{a^{k+2}-1}{a-1} \]
\end{proof}

\begin{proof}{2: Ohne Induktion}
\[S_n:=\sum_{j=0}^n a^j\]
\[\Longrightarrow a*S_n=a*\sum_{j=0}^n a^j = \sum_{j=0}^n a*a^j = \sum_{j=0}^n a^{j+1} = \sum_{j=1}^{n+1} a^j\]
\[\Longrightarrow a*S_n-S_n=\sum_{j=1}^{n+1} a^j -\sum_{j=0}^{n+1} a^j = a^{n+1}-a^0=a^{n+1}+1\]
\[\Longrightarrow (a-1)S_n =a^{n+1}+1 \Longrightarrow S_n = \frac{a^{n+1}+1}{a-1}\]

\end{proof}

\end{example}

\subsubsection{Notation: Aussagen}

Seien $A,B,C,D$ mathematische Aussagen 

\textbf{Syntax}

\begin{itemize}
  \item $\lnot A$: nicht A
  \item $A \wedge B$: A und B
  \item $A \vee B$: A oder B
  \item $A \Longrightarrow B$: A impliziert B, aus A folgt B
  \item $A \Longleftrightarrow $: A äquivalent zu B, A genau dann, wenn B
\end{itemize}

\begin{example}
\begin{itemize}
  \item $(A \Longleftrightarrow B) \Longleftrightarrow ((A\Longrightarrow B)\wedge(B \Longrightarrow A)) $
  \item $(A \Longrightarrow B) \Longleftrightarrow (\lnot B \Longrightarrow \lnot A)$
\end{itemize}
\end{example}

\subsubsection{Quantoren}

Oft enthalten Aussagen eine freie Variable

\begin{example}
\begin{itemize}
  \item $A(x):x$ ist eine Primzahl
  \item $A(n): \sum_{j=1}^n j=\frac{n(n+1)}{2}$
\end{itemize}
\end{example}

Dann gehört eine Grundmenge $U$, sodass $A(x)$ eine mathematische Aussage ist von $x \in U$

\textbf{Syntax:}

\begin{itemize}
  \item $\exists$ es gibt
  \item $\forall$ für alle
  \item $\exists x \in U:A(x):$ es gibt ein Element $x \in U$, sodass $A(x)$ wahr ist.
  \item $\forall x \in U:A(x): A(x)$ ist wahr für alle $x$.
\end{itemize}

\subsection{Wohlordnungsprinzip für $\mathbb{N}$}

Wir wollen beweisen $\forall n \in \mathbb{N}: A(x)$ wahr ist

\textbf{Negation:}
\[\lnot(\forall n \in \mathbb{N}: A(x))=\exists n \in \mathbb{N}: \lnot A(x)\]
\[\lnot(\exists n \in \mathbb{N}: \lnot A(x))=\forall n \in \mathbb{N}: \lnot(\lnot A(x)) = A(x)\]

\textbf{Also:} $G=\{n\in \mathbb{N}: \lnot A(n)\}$ müssen zeigen, dass $G=\emptyset$

\subsubsection{Satz 6}

Sei $A \subset \mathbb{N}, A \neq \emptyset$, dann hat $A$ ein kleinstes Element!

D.h. $\exists n_0 \in A$ mit $\forall k \in A: k \geq n_0$

\subsubsection{Satz 7}

$\sqrt{2}$ ist nicht rational.

\textbf{Angenommen:} $\sqrt{2}$ ist rational $\Longrightarrow \exists m \in \mathbb{Z}, n \in \mathbb{N}, \sqrt{2}=\frac{m}{n}$

$G:=\left\lbrace n \in \mathbb{N}: \exists m \in \mathbb{Z}: \sqrt{2} = \frac{m}{n} \right\rbrace \subset \mathbb{N}$

\textbf{Wollen:} $G=\subset$

\textbf{Angenommen:} $G \neq \emptyset \Longrightarrow G$ hat ein kleinstes Element (Satz 6)

$\sqrt{2}=\frac{m}{n_0}:$ dann ist $m-n_0 = (\sqrt{2}-1)n_0 \Longrightarrow 0<m-n_0<n_0$ also $m-n_0 \in \mathbb{N}$

$\Longrightarrow \sqrt{2} = \frac{m}{n_0} = \frac{m(m-n_0)}{n_0(m-n_0)} = \frac{m^2-m*n_0}{n_0(m-n_0)} = \frac{2n_0^2-m*n_0}{n_0(m-n_0} = \frac{2n_0-m}{m-n_0}$

Also hat $G$ kein kleinstes Element $\Longrightarrow G = \emptyset$ 

\subsubsection{Satz 8}

$K \in \mathbb{N}$, damit $\sqrt{k} \subset \mathbb{N}$ oder irrational

\begin{proof}

\textbf{Negation:} $\sqrt{k} \notin \mathbb{N}$ und $\sqrt{k}$ ist rational

\textbf{Annahme:} $\sqrt{k} \in G \backslash \mathbb{N}$

$G:=\left\lbrace n \in \mathbb{N}: \exists m \in \mathbb{Z}: \sqrt{k} = \frac{m}{n} \right\rbrace \subset \mathbb{N}$

\textbf{Wollen:} $G = \emptyset$!

\textbf{Angenommen} $G \neq \emptyset$. Sei $n_0$ kleinstes Element in $G$

$\sqrt{k} = \frac{m}{n_0} = \frac{m(m-n_0)}{n_0(m-n_0)} = \frac{m^2-m*n_0}{n_0(m-n_0)} = \frac{k*n_0^2-m*n_0}{n_0(m-n_0} = \frac{k*n_0-m}{m-n_0}$ 

$\Longrightarrow k>1$

Für Widerspruch brauchen wir:

$0<m-n_0<n_0$

$m-n_0 = \sqrt{k}*n_0-n_0=(\sqrt{k}-1)n_0>0,\sqrt{k}>1$ 

$m-n_0 = (\sqrt{k}-1)n_0 < n_0$

D.h. $\sqrt{k} -1<1 \Longrightarrow \sqrt{k}<2 \Longrightarrow k<4$

$k \leq 3 \Longrightarrow $ (Bullshit)

Versuchen mal $m-l*n_0,l \in \mathbb{N}$ geeignet

$\sqrt{k} = \frac{m}{n_0} = \frac{m(m-l*n_0)}{n(m-l*n_0)} = \frac{k*n_0-l*n_0}{n(m-l*n_0)}, k*n_0-l \in \mathbb{Z}$

\textbf{Brauchen:}  $0<m-l*n_0<n_0 \Longleftrightarrow 0<(\sqrt{k}-l)n_0<n_0$ 

\textbf{Brauchen:}  $0<\sqrt{k}-l<1$, wähle $l \in \mathbb{Z}$, sodass $l < \sqrt{k}<l+1$

sollte möglich sein, falls $\sqrt{k} \notin \mathbb{N}$ 

\end{proof}

\subsection{Körper- und Anordnungsaxiomen}

\begin{example}

$0$ ist eindeutig!
Sei $0'$ auch neutrales Element der Addition
\begin{align*}
  \Longrightarrow 0 = \text{ }& 0' = 0\\
                      & 0  = 0+0'=0'+0=0'\\
                      & 0' + 0=0'
\end{align*}


\end{example}

\pagebreak

\begin{example}

$a+x=b$ hat eine eindeutige Lösung

$x=b+(-a)=b-a$

\begin{align*}
  \text{Sei } a+x=b & \Longrightarrow (-a)+(a+x)=(-a)+b\\
                    & \Longrightarrow ((-a)+a)+x=b+(-a)\\
                    & \Longrightarrow 0 + x = b+ (-a)
\end{align*}

Wenn $x=b+(-a)$

\begin{align*}
  \Longrightarrow \text{ } & a+x=a+(b+(-a))=b+((-a)+a)\\
                           & = b+(a+(-a))\\
                           & =b+0=b
\end{align*}

\end{example}

In jedem Körper gilt:

\[\frac{a}{c}+\frac{b}{d}=\frac{ad+bc}{cd}\]
\[\frac{a}{c}*\frac{b}{d}=\frac{ab}{cd}\]
\[\frac{\frac{a}{c}}{\frac{b}{d}}=\frac{ad}{bc}\]


TODO: Handout über Körperaxiome muss hier rein!

\subsubsection{Satz 13}

Sei $\mathbb{K}$ ein angeordneter Körper, $a,b,c,d,x,y \in \mathbb{K}$ Dann gilt:

\begin{enumerate}
  \item $a>b \Longleftrightarrow a-b>0$
  \item $a>b \wedge c>b \Longrightarrow a+c>b+a$
  \item $a>0 \wedge x > y \Longrightarrow ax>ay$
  \item $a>0 \Longleftrightarrow -a<0$
  \item Vorzeichenregeln:
  \begin{enumerate}
    \item $x>0;y<0 \Longrightarrow xy <0$
    \item $a<0; x>y \Longrightarrow ax < ay$
  \end{enumerate}
\end{enumerate}

\pagebreak

\begin{proof}

\begin{enumerate}
  \item Sei $a>b \Longrightarrow a-b=a+(-b)>b+(-b)=0$  \\
    Sei $a-b>0 \stackrel{(O4)}{\Longrightarrow} a= b+(a-b)>b$
  \item Sei $a>b,c>d \stackrel{(O4)}{\Longrightarrow} a+c>b+d$ und \\
    $b+c>b+d \stackrel{(O1)}{\Longrightarrow} a+c>b+d$
  \item Sei $a>0,x>y \stackrel{(1.)}{\Longrightarrow} x-y>0 \stackrel{(O5)}{\Longrightarrow} a(x-y)>0$ \\
    $\Longrightarrow ax-ay >0$
    $\Longrightarrow ax >ay$
  \item Aus $a>0 \stackrel{(O4)}{\Longrightarrow} (-a)=(-a)+0<(-a)+a=0$  \\
    Aus $a<0 \stackrel{(O4)}{\Longrightarrow} (-a)+a<0+a=a$
  \item Folgt aus (4) und (O5)
\end{enumerate}

$\Longrightarrow$ fertig.

\end{proof}

\subsubsection{Satz 14}

Sei $(\mathbb{K},+,*)$ ein angeordneter Körper $\Longrightarrow$

\begin{enumerate}
  \item $a \neq 0 \Longrightarrow a^2>0$ insbesondere $1>0$
  \item $a > 0 \Longrightarrow \frac{1}{a}>0$
  \item $a > b > 0 \Longrightarrow \frac{1}{a}<\frac{1}{b}$ und $\frac{a}{b}>1$
\end{enumerate}

\begin{proof}

\begin{enumerate}
  \item $a^2=a*a$ \\
    aus $a>0 \stackrel{(O5)}{\Longrightarrow} a^2=a*a>0$ \\
    aus $a<0 \stackrel{(S15(5))}{\Longrightarrow} a*a>0$
  \item Sei $a \neq 0 \Longrightarrow a*\frac{1}{a} = 1 > 0 \stackrel{(S1(5))}{\Longrightarrow} a>0 \wedge \frac{1}{a}>0$ \\
    oder $a<0 \wedge \frac{1}{a}>0$
  \item Sei $a>b>0 \stackrel{(2)}{\Longrightarrow} \frac{1}{a}>0;\frac{1}{b}>0;a*b>0;a-b>0 (S13(1))$ \\
  $\Longrightarrow \frac{1}{b} - \frac{1}{a} = \frac{1}{b}(a-b)\frac{1}{a}=(a-b)\frac{1}{b}*\frac{1}{a}>0$
\end{enumerate}

fertig

\end{proof}

\textbf{Vorliegende Definition:} Die $\mathbb{R}$ sind ein geordneter Körper (da fehlt noch was)

\subsubsection{Absolutbetrag}

\[
  |x| = \left\{ 
    \begin{array}{rl}
       x, & \text{ falls } x>0 \\
       0, & \text{ falls } x=0\\
      -x, & \text{ falls } x<0
    \end{array}\right.
\]

\subsubsection{Signumfunktion / Vorzeichenfunktion}

\[
  sign(x) = \left\{ 
    \begin{array}{rl}
       1, & \text{ falls } x>0 \\
       0, & \text{ falls } x=0\\
      -1, & \text{ falls } x<0
    \end{array}\right.
\]
\subsubsection{Min- und Max-Funktion}

\[
  max(x,y) = \left\{ 
    \begin{array}{rl}
       x, & \text{ falls } x>y \\
       y, & \text{ falls } y\geq x
    \end{array}\right.
\]
\[
  min(x,y) = \left\{ 
    \begin{array}{rl}
       x, & \text{ falls } x<y \\
       y, & \text{ falls } y\leq x
    \end{array}\right.
\]

\subsubsection{Folgerungen}

\begin{enumerate}
  \item $\forall x \in \mathbb{R};x=|x|sgn(x)$ \\
    $|-x|=|x|;x \leq |x|$
  \item $\forall x \neq 0: |x|>0$
  \item $\forall x,y \in \mathbb{R}: |x*y|=|x|*|y|$ \\
    $sgn(x*y)=sgn(x)*sgn(y)$
  \item $\forall x \in \mathbb{R}, \forall e >0$ \\
    hat $|x-a|<e \Longleftrightarrow a-e<x<a+e$ \\
    insbesondere $|x|<e \Longleftrightarrow -e<x<e$
  \item TODO: Stimmt das so? $|x|=max(x,-x)$ \\
        Beweis: einfach
\end{enumerate}

\subsubsection{Satz 15: Dreiecksungleichung}

\begin{align*}
  \forall a,b \in \mathbb{R}: & |a+b| \leq |a|+|b|\\
                              & ||a|-|b||\leq|a-b|
\end{align*}

\begin{proof}

\[\text{Falls } a+b \geq 0 \Longrightarrow |a+b|=a+b \leq |a|+b\leq|a|+|b|\]

\[\text{Falls } a+b < 0 \Longrightarrow -(a+b)>0 \Longrightarrow |a+b|=-(a+b)\]
\[=(-a)+(-b)\leq |-a|+(-b)\leq |-a|+ |-b|=|a|+|b|\]
\[|a|=|(a-b)+b|\leq|a-b|+|b|\Longrightarrow |a|-|b|\leq|a-b|\]

Vertausche a und b
\[|b|-|a|\leq |b-a|=|-(a-b)|=|a-b|=-(|a|-|b|)\]
\[\Longrightarrow||a|-|b||=max(|a|-|b|,-(|a|-|b|)\leq |a-b|\]
fertig

\end{proof}

\subsubsection{Satz 16: Abstandsungleichung}

$\forall a,b,c \in \mathbb{R}: d(a,c) \leq d(a,b)+d(b,c)$

\begin{proof}

\[d(a,c)=|a-c|=|(a-b)+(b-c)| \leq |a-b|+|b-c|\]
\[=d(a,b)+d(b,c)\]
fertig

\end{proof}

\subsection{Obere und untere Schranken, Supremum und Infimum}

\subsubsection{Obere und Untere Schranken}

Sei $A \subset \mathbb{K}$, $\mathbb{K}$ ein geordneter Körper.

A heißt nach oben beschränkt falls $ \exists \alpha \in \mathbb{K}, \forall a \in A : a \leq \alpha$.

\textbf{Schreiben} $A \leq \alpha$. $\alpha$ heißt obere Schranke von $A$.

$A$ heißt nach unten beschränkt falls $\exists \beta \in \mathbb{K},\forall a \in A: \beta \leq a$

\textbf{Schreiben} $\beta \leq A$. $\beta$ heißt untere Schranke von $A$

\subsubsection{Maximum und Minimum}

$A$ heißt maximales Element (oder Maximum) von A, falls $\alpha$ obere 
Schranke für $A$ ist und $\alpha \in A$

$A$ heißt minimales Element (oder Minimum) von A, falls $\beta$ untere Schranke für $A$ ist und $\beta \in A$

\begin{proof}
Falls Maximum existiert, dann ist es eindeutig. Genauso für das Minimum. 

B. \quad H.A %Year .... WTF% 

$A=\{x \in \mathbb{R}, x>0\}, \inf(A)=0$

nach unten beschränkt, hat sein Minimum.

$B=\{x:x<0\}, \sup(B)=0$

\end{proof}

\subsubsection{Definition 18: Supremum, Infimum}

$A \subset \mathbb{R}, A \neq \emptyset$

$\sup(A)=\sup A := $ kleinste obere Schranke von $A$

$\inf(A)=\inf A := $ kleinste obere Schranke von $A$

\subsubsection{Lemma 19}

Sei \textbf{$\alpha$ eine obere Schranke} für $A \neq \emptyset$. Dann gilt

\[\alpha = \sup(A) \Longleftrightarrow \forall \epsilon > 0 \exists a_{\epsilon} \in A : \alpha - \epsilon < a_{\epsilon} \text{\quad(oder) } \alpha - \epsilon \leq a_{\epsilon}\]

\begin{proof}

Sei $\alpha = \sup(A)$ und $\epsilon > 0 \Longrightarrow \alpha - \epsilon$ ist keine obere Schranke für $A$.

Also $\exists a_{\epsilon} \in A: \alpha -\epsilon < a_e\surd$

\qq{$\Longleftarrow$} Beweis durch Kontraposition.

N.B.: $(E \Longrightarrow F) \Longleftrightarrow (\lnot F \Longrightarrow \lnot E)$
\[\lnot (\alpha = \sup(A))= \alpha > \sup(A) \]
\[\lnot (\forall \epsilon > 0 \exists a_{\epsilon} \in A:\alpha -\epsilon < a_{\epsilon}) \]
\[\exists \epsilon > 0 \boxed{\forall a_{\epsilon} \in A:\alpha -\epsilon \geq a_{\epsilon}}\]

\textbf{Annahme:} $\alpha > \sup(A)$

\textbf{Wählen:} $\epsilon:=\alpha-\sup(A)$

\textbf{Damit gilt:} $\forall a \in A : a \leq \sup(A) = \alpha - \epsilon$  

\end{proof}

\subsubsection{Definition 20: Vollständigkeitsaxiom}

Die reellen Zahlen $\mathbb{R}$ sind der angeordnete Körper in dem jede nicht leere Menge die nach oben beschränkt ist ein Supremum hat.

Oder: $\mathbb{R}$ ist der ordnungsvollständige Körper.

\begin{example}
\[\sup(\{x \in \mathbb{R}, x < 0\}) = 0\]
\[\sup(\{x \in \mathbb{R}, x^2 < 0\}) \text{ hat ein Suprenum (später: das Suprenum ist } \sqrt{2} \text{)}\]
\end{example}

\subsubsection{Die Menge $\bar{\mathbb{R}}$}

Die Menge $\bar{\mathbb{R}}:= \mathbb{R} \cup \{\infty\} \cup \{-\infty\}$ erweitert die Zahlengerade

\textbf{Es gilt:} $-\infty < x < \infty \forall x \in \mathbb{R}$

\textbf{Regeln:}

\begin{itemize}
  \item $\infty + x := \infty$
  \item $-\infty + x := -\infty$
  \item $\infty * x := \infty, \quad x>0$
  \item $\infty * x := -\infty, \quad x<0$
  \item $\frac{x}{\infty}:=0=\frac{x}{-\infty}$
  \item $\infty + \infty := \infty$
  \item $-\infty - \infty := -\infty$
  \item $\infty * \infty := \infty$
  \item $\infty * (-\infty) := -\infty$
\end{itemize}

\textbf{Nicht definiert:}

\begin{itemize}
  \item $\infty-\infty$
  \item $0*\infty$
\end{itemize}

\subsubsection{Intervalle}

\begin{itemize}
  \item $a \leq b \quad [a,b] := \{x \in \mathbb{R}:a \leq x \leq b\}$ abgeschlossenes Intervall
  \item $a \leq b \quad (a,b) := \{x \in \mathbb{R}:a < x < b\}$ offenes Intervall
  \item $[a,b) := \{x \in \mathbb{R}:a \leq x < b\}$ rechts halboffenes Intervall
  \item $(a,b]:= \{x \in \mathbb{R}:a < x \leq b\}$ links halboffenes Intervall
  \item $(-\infty,a]:= \{x \in \mathbb{R}:x \leq a\}$
  \item $(-\infty,a):= \{x \in \mathbb{R}:x < a\}$
  \item $[a,\infty):= \{x \in \mathbb{R}:x \geq a\}$
  \item $(a,\infty):= \{x \in \mathbb{R}:x > a\}$
\end{itemize}

\begin{proof}
$\sup([a,b])=\sup([a,b)) = b,$ falls $a<b \quad $

Wenn eine Menge $A$ ein Maximum hat 

$\Longrightarrow$ Supermum ist gleich dem Maximum

\end{proof}

\subsubsection{Supremum und Infimum der leeren Menge}
\textbf{Setzen:}
\[\sup(\emptyset):=-\infty\]
\[\inf(\emptyset):=+\infty\]

\subsection{Definition von $\mathbb{N}$ als Teilmenge von $\mathbb{R}$}

\subsubsection{Definition 21}

Eine Menge $A \subset A \mathbb{R}$ heißt \underline{induktiv} falls:

\begin{enumerate}
  \item $1 \in A$
  \item Falls $k \in A$, dann ist $k+1 \in A$
\end{enumerate}

\begin{example}

$A= [1,\infty)$ ist induktiv.

$A:= \{1\} \cup [1+1,\infty)$ ist induktiv
\end{example}

$\mathbb{N}:=$ kleinste induktive Teilmenge von $\mathbb{R}$
\[:= \bigcap_{A \text{ist induktiv}} A \qquad\qquad \text{nachdem dass diese Menge induktiv ist}\]

\subsubsection{Satz 21: Induktionsprinzip}

Ist $M \subset \mathbb{N}$, mit 

\begin{enumerate}
  \item $1 \in M$
  \item Aus $k \in M$ folgt $k+1 \in M$
\end{enumerate}

$\Longleftrightarrow M = N$

\subsubsection{Satz 22}
\begin{enumerate}[1)]
\item $\forall n \in \mathbb{N}: n \geq 1$ oder $n \leq 1 + 1$ und $n = 1$ oder $n-1 \in \mathbb{N}$
\item $\forall n,m \in \mathbb{N}: n+m \in \mathbb{N}$ und $n*m \in \mathbb{N}$
\item $\forall n,m \in \mathbb{N} n \geq m \implies n-m \in \mathbb{N}_0 = \mathbb{N} \cup \{0\}$
\item Sei $n \in \mathbb{N}$ Dann existiert kein $m \in \mathbb{N}$ mit $n < m < n+1$
\item Sei $A \subset \mathbb{N}: A \neq \emptyset \implies A$ hat ein kleinstes Element
\end{enumerate}

\begin{proof}
{Sei $\tilde{A} = \{1\} \cup [2, \infty)$ ist induktiv $\implies \mathbb{N} \subset B \implies n = 1$ oder $n \geq 2$}
\begin{description}
\item[$a_1$)] $1 \in A:$ klar
\item[$a_2$)] $1 + 1 \in A:$ klar
\item[$b$  )] Sei $k \in A, k \neq 1 \implies 1 \leq k - 1 \in \mathbb{N}$\\
folgt $1 + 1 \leq (k - 1) + 1 = k \in \mathbb{N}$\\und $(k + 1) - 1 = k \geq 1 + 1 \geq 1 \implies k + 1 \in A$\\ $\implies A \subset \mathbb{N}$ ist induktiv $\implies A = \mathbb{N} \implies \underline{1)}$ \\
\end{description}
$B := \{n \in \mathbb{N}:$ für $m \in \mathbb{N}$ mit $m \leq n \implies n-m \in \mathbb{N}_0$
\begin{description}
\item[$a$  )] $1 \in B$, da $m \in \mathbb{N}$ und $m \leq 1 \underbrace{\implies}_{1)} m = 1 \implies n-m = 1-1 = 0$
\item[$b$  )] Sei $k \in B$ und $m \in \mathbb{N}$ mit $m \leq k + 1$ \\Falls $m = 1 \implies (k + 1) - 1 = k \in \mathbb{N} \implies k + 1 \in B$\\ Falls $1 < m \in \mathbb{N} \implies m - 1 \in \mathbb{N}$ (da $A = \mathbb{N}$)\\ $\implies \mathbb{N}_0 \ni k - (m - 1) = (k + 1) - m \implies k + 1 \in B$\\$\implies B$ ist induktiv $\implies B = \mathbb{N} \implies \underline{3)}$
\item[2)] Gegeben: $m \in \mathbb{N}: C:=\{n \in \mathbb{N} | n + m \in \mathbb{N}\}$\\Zeige C ist induktiv!\\Für $m * n $ analog
\item[4)] Aus $n,m \in \mathbb{N} und n < m < n + 1$\\$\implies 0 < \underbrace{m-n}_{\in \mathbb{N}\texttt{ nach 3)}}<1$ (Widerspruch! zu 1))
\item[5)] Sei $M \subset \mathbb{N}$, ohne ein kleinstes Element\\$\implies 1$ ist kleinste Element von $\mathbb{N} \implies 1 \notin M$
\end{description}
$D:=\{n \in \mathbb{N}: n < M\} = \{n \in \mathbb{N}: \forall m \in M: n < m\}$\\Wissen:
\begin{description}
\item[a)] $1 \in D$
\item[b)] Sei $k \in D$ d.h. $k < m \forall m \in M$\\$\implies D$ ist induktiv$\implies D = \mathbb{N} \implies M \subset \mathbb{N} \backslash D = \mathbb{N} \backslash M = \emptyset$ (q.ed)
\end{description}
\end{proof}
\subsubsection{Satz 23} $\mathbb{R}$ ist Archimedisch angeordnet $\mathbb{N} \subset \mathbb{R}$ ist \underline{nicht} nach oben beschränkt\\ insbesondere $\forall a > 0, b \in \mathbb{R} \exists n \in \mathbb{N}: n * a > b$

\begin{proof}
\paragraph*{Angenommen} $\mathbb{N}$ ist nach oben beschränkt $\underbrace{\implies}_{vollst. Axiom} a = Sup\mathbb{N}\in\mathbb{R}$\\$\implies \alpha - 1$ ist keine obere Schranke für $\mathbb{N}$\\$\implies \exists n \in \mathbb{N}, n > \alpha - 1 \iff  \underbrace{n+1}_{\in \mathbb{N}} > \alpha$ (Widerspruch!)
\subparagraph{Wähle}
$x = \frac{b}{a} \in \mathbb{R} \implies \exists n \in \mathbb{N}: n > x = \frac{b}{a} \underbrace{\implies}_{a > 0} n* a > b$ (q.ed)
\end{proof}
\subsection{Grenze und rationale Zahlen}
$\mathbb{Z} := \mathbb{N}_0 \land (-\mathbb{N}), -\mathbb{N} := \{-n, n \in \mathbb{N}\}$\\$\mathbb{Q}=\{\frac{m}{n}: m\in\mathbb{Z},n \in\mathbb{N}\}$
\subsubsection{Satz 24}
($\mathbb{Z}, +, *$) ist ein kommutativer Ring mit Eins, d.h. alle Körperaxiome sind erfüllt. Aber es gibt kein inverses Element der Multiplikation. ($\mathbb{Q}, +, *$) ist ein angeordneter Körper.

\begin{proof}
	Nachrechnen
\end{proof}
\paragraph{Notation}$\mathbb{Z}_p := \{ m\in\mathbb{Z}: m\geq p\}$\\$p \in \mathbb{Z} := p + \mathbb{N}_0$
\paragraph{Alle} $k \mapsto k + p - 1$ bildet $\mathbb{N}$ bijektiv auf $\mathbb{Z}_p$ ab. \\$\Rightarrow$ Alle Eigenschaften von $\mathbb{N}$ gelten auch für $\mathbb{Z}_p \forall p \in \mathbb{Z}$\\$\Rightarrow$ Lemma 25: Jede nach unten bzw. oben beschränkte Teilmenge $\neq \emptyset$ von $\mathbb{Z}$ besitzt ein Minumum bzw. ein Maximum
\subsubsection{Korollar 26}
\begin{enumerate}[1)]
\item Seien $x,y \in \mathbb{R}, y*x > 1$\\$\implies m\in\mathbb{Z}, x < m < y$
\item ($\mathbb{Q}$ ist dicht in $\mathbb{R}$) Seien $x,y \in \mathbb{R}, x < y \implies \exists r \in \mathbb{Q}: x < r <y$
\end{enumerate}
\paragraph{Beweis} 
\begin{enumerate}[1)]
\item Sei $y - x > 1, A := \{ m \in \mathbb{Z}: m > y\} \neq \emptyset$\\$\implies$ Sei $n_0 = min(A)$ existiert $\in \mathbb{Z}$\\$\implies n_0 \in A: n_0 \geq y$ und $n_0 -1 < y$\\$m:=n_0 -1 \in \mathbb{Z}$ und $m + 1 \geq y, n <y$\\$\implies m \geq y -1 > x \implies x < m < y$
\item Sei $x,y \in \mathbb{R}: x < y \iff a: -y -x > 0$\\S.23 $\implies \exists n \in \mathbb{N}: n * a > 1 \iff n * x - n * y > 1$\\$\implies \exists m \in \mathbb{Z}: n*x<m<n*y \iff x <\frac{m}{n} < y$
\end{enumerate}
\subsection{Endliche und abzählbare Mengen}
\subsubsection{Definition 27 (Cantor)} A, B Mengen heissen gleichmächtig (oder äquivalent) $A \sim B$, falls es eine Bijektion $f: A \rightarrow B$ gibt.\\B heisst mächtiger als A, $|A| \leq |B|$, falls es eine Injektion $f: A \rightarrow B$ gibt.
\paragraph{Bemerkung}
\begin{enumerate}[1)]
\item $A \sim B$ ist eine äquivalenzrelation, d.h. reflexiv ($A \sim A$), symmetrisch ($A \sim B \implies B \sim A$) und transitiv ($A \sim B, B \sim C \implies A \sim C$)
\item $A \leq \mathbb{R} \iff \exists$ Surjektion $h: B \rightarrow B$
\item (Cantor) Bernsten-Schröder-Theorie $|A| \leq |B|$ und $|B| \leq |A| \iff A \sim B$
\end{enumerate}
\subsubsection{Definition 28} Sei $n \in \mathbb{N}_0 [0] := \emptyset$ und rekrusiv $[n + 1] = [n] \cup [n+1]$\\($\implies([n]:=\{k \in \mathbb{N}: 1\leq k \leq n \}$)\\Eine Menge A heisst endlich, falls $\exists n \in \mathbb{N}_0$ mit $A \sim [n]$, sage A hat n Elemente $card(A) := n$ (Kardinalität)\\$card\emptyset = 0$ Eine Menge A ist unendlich, falls sie nicht endlich ist.\\\\A heisst abzählbar (abzählbar unendlich), falls $A \sim \mathbb{N}$\\A ist höchstens abzählbar, falls A endlich ist oder abzählbar ist, ansonsten heisst sie überabzählbar.
\paragraph{Beweis}
\begin{enumerate}[1)]
\item A höchstens abzählbar $\iff \exists$ Surjektion $f: \mathbb{N} \rightarrow A$
\item Unendliche Mengen sind tricky\\$G = \{ n \in \mathbb{N}: $n ist gerade$\} = \{2*n: n \in \mathbb{N}\}$\\$f:\mathbb{N}\rightarrow G, n \mapsto 2n$ ist bijektiv, d.h. $\mathbb{N} \sim G$
\item Hilkerts Hotel
\item $[0,1]\sim [0,1)$
\subparagraph{Beweis} Konstruieren $f: [0,1] \rightarrow [0,1)$\\Für $x \in [0,1]\backslash (\bigcup_{n\in\mathbb{N}} \{\frac{1}{n}\}):f(x)=x$\\$n\in\mathbb{N}:f(\frac{1}{n}):=\frac{1}{n+1}$ Rechne nach f ist bijektiv!
\end{enumerate}
\subsubsection{Satz 29}
\begin{enumerate}[1)]
\item $A\sim [n], A\sim [m] \implies n = m$ (d.h. Kardinalität ist eindeutig)
\item ist $A \in B, B endlich \implies A endlich$
\item $A, B$ endlich und desjunkt $\implies card(A\cup B) = (cardA + cardB)$
\end{enumerate}
\paragraph{Beweis}
\begin{enumerate}[1)]
\item $\implies [n] \sim [m]$ durch Induktion $\implies n = m$\\Fall $n = 1$ (CHECK!)\\$n\rightarrow n + 1$: IA $\tilde{\phi}:[n]\rightarrow [m]$\\bijektiv $\implies n = m$\item Sei $\phi:[n+1]\rightarrow[m+1]$ Bijektion:\\Durch Vertauschen von 2 Elementen kann man erreichen, dass $\phi (n + 1) = m+1 \implies \phi |_{[n]}:[n]\rightarrow [n]$ bijektiv $\implies n = m \implies  m + 1 = m$ (WTF?) (q.ed)
\item Beweis der Induktion: einfach.
\item Sei $A \sim [n], b \sim [m] \implies B \sim m + [n] := \{ k \in \mathbb{N}: n + 1 \leq k \ leq m + n \} \implies A \cup B \sim [n] \cup (m + [n]) = [n + m]$
\end{enumerate}
\paragraph{Lemma 30} Jede endliche teilmenge von $\mathbb{R}$ hat ein Minimum und ein Maximum

\begin{proof}
$A = \{a_1\}$\\Ist $A = \{a_1, a_{n+1}\}$ und $C:= min\{a_1, a_n\} \implies minA = min(C, a_{n+1})$
\end{proof}
\subsubsection{Satz 31}
\begin{enumerate}[1)]
\item Ist $A<B,B$ höchstens abzählbar $\implies A$ höchstens abzählbar
\item Jede unendliche Menge besitzt eine abzählbare Teilmenge
\item A, B abzählbar $\implies A \times B$ abzählbar\\ insbesondere $\mathbb{N} \times \mathbb{N}$ abzählbar
\item Sei $\{A_k\}$ eine höchstens abzählbare Menge von Menge $A_3, A_2$ höchstens abzählber $\implies$ $\bigcap_k$ $A_k$ ist höchstens abzählbar
\end{enumerate}

\begin{proof}
\begin{enumerate}[1)]
\item O.B.d.A $B = \mathbb{N}$, also $A \subset \mathbb{N}$\\$\implies A$ hat ein kleinstes Element $a_1$\\$\implies A \{a_1\}$ hat ein kleinnstes Element $a_2$\\usw...\\ist $A_n = \emptyset \implies A$ ist endlich, ansonsten $A = \{a_1, a_2, a_3, ...\}$\\Bijektion $f:\mathbb{N}\rightarrow A, n\mapsto a_n \implies A$ ist abzählbar
\item ist $A$ unendlich $\implies$ wähle $a_1 \in A$\\
	$a_2 \in A \backslash \{a_1\} =: A_1$
	induktiv $a_{n+1} \in A_n := A_{n+1} \backslash \{a_n\}$\\
	$\implies \{a_1, a_2, ...\}$ abzählbar
\item Da $A \sim \mathbb{N}, B \sim \mathbb{N} \implies$ reicht zu zeigen $\mathbb{N} \times \mathbb{N}$ ist abzählbar, da $\mathbb{N} \times \mathbb{N}$ unendlich ist $\implies$ zu zeigen $\mathbb{N} \times \mathbb{N}$
\item ist höchstens abzählbar\\
	$\phi (m,n) = 2^m * 3^n$\\
	$\phi : \mathbb{N} \times \mathbb{N} \implies \mathbb{N}$ ist injektiv\\
	In der Tat: Sei $\phi (m,n) = \phi (p,q)$\\
	d.h. $2^m * 3^n = 2^p * 3^q$\\
	o.B.d.A $p \geq m$\\
	$\implies 3^n = 2^{p-m} * 3^q$\\
	$\implies p = m$\\
	$\implies n = q$\\
\item Schreiben $A_k = \{a_{kn} :\underbrace{1\leq n \leq P_k}_{endlich}, P_k \in \mathbb{N}$ oder $\underbrace{1 \leq n \in \mathbb{N}}_{unendlich}\}$\\
	Falls $A_k$ paarweise disjunkt sind. Dann erzeugt diese Nummerierung von $A_k$ eine Injektion.
	$$a_{kn} \mapsto (kn) \textbf{ von } A = \bigcup_{k \in I} A_k \rightarrow \mathbb{N} \times \mathbb{N} \leftarrow \textbf{abzählbar}$$\\
	sind $A_k, k \in I$ nicht paarweise disjunkt:\\
	$B_1 = A_1, B_2 = A_2 \backslash A_1,$\\
	$B_{n+1} = A_{n+1} \backslash \{A_1 \cup A_2 \cup ... \cup A_n\}$\\
	$\implies B_k$ sind paarweise disjunkt und höchstens abzählbar\\
	$\implies \bigcup_{k} A_k$ ist höchstens abzählbar
\end{enumerate}
\end{proof}
\subsubsection{Korollar 32} $\mathbb{G}$ ist abzählbar\\
\begin{proof}
$\mathbb{G} = \{ \frac{m}{n}: m \in \mathbb{Z}, n \in \mathbb{N}\}$ \qq C $\{(m,n), m \in \mathbb{Z}, n \in \mathbb{N}\}$
\end{proof}
\paragraph{Bemerkung} Es gibt eine explizite Abbildung von $\mathbb{G}$ mittels eines Baumes. Literatur: Neil Calkin, Herbert Will: Recounting the Rationals
\subsubsection{Satz 33} A enthalte mindestens 2 Elemente, $\implies A^\mathbb{N} = \{ f: \mathbb{N} \rightarrow A\}$ überabzählbar
\subsubsection{Lemma 34 (Cantor)} Sei A meinge $\implies$ existiert \underline{keine} surjektive Abbildung $f: A \rightarrow P(A)$
\begin{proof}
	Sei $f: A \rightarrow P(A)$\\
	d.h. $\forall x \in A: 2(x) \subset A$\\
	$B:=\{ x \in A: x \notin f(x)\} \subset A$\\
	wäre f surjektiv\\
	$\implies \exists x \in A, f(x) = B$\\
	\begin{enumerate}[1. {Fall}:]
		\item $x \in B = f(x) \implies x \notin f(x)$ (WIDERSPRUCH!)
		\item $x \notin B = f(x) \implies x \in B = f(x)$ (WIDERSPRUCH!)\\
			$\implies$ f ist nicht surjektiv!
	\end{enumerate}
\end{proof}
\subsubsection{Korollar 36} Sei $I := [a,b]$, oder $(a,b) \subset \mathbb{R}$\\
$a < b> \implies I$ ist überabzählbar

\begin{proof}
Skalieren $\implies$ o.B.d.A. $a = o, b = 1$ zu $f \in \{0,1\}^\mathbb{N}$
\end{proof}
\paragraph{Dezimalbruchentwicklung:} $$x_f := \sum_{n = 1}^\infty f(n) * 10^{-n} \in [0,1]$$
beachte: $f_1 + f_2 \implies xf_1 + xf_2$
\subsection{Einfache Folgerung aus Induktion}
\subsubsection{Satz 37 (...}
$\forall x \in \mathbb{N}, x > -1 (1+x)^n \geq 1 + nx$ und Ungleichung ist strikt (d.h. > gilt, falls $n \geq 2, x \neq 0$)

\begin{proof}
	IA $n = 0 | (1 + x)^0 = 1 + 0x$\\
	Im Ange. gilt: $(1 + x)^k \geq 1 + k x$\\
	$implies (1 + x)^{k + 1} = (1 + x)^k * \underbrace{(1 + x)}_{> 0} \geq (1 + kx) (1 + x)$\\
	$= 1 + (k + 1)x = 1 + (k + 1)x + k x^2 \geq 1 + (k + 1)x$
\end{proof}
\subsubsection{Definition 38} 0! = 1\\
	$n \in \mathbb{N}_0 | (n + 1)! := ...$\\
	d.h. $n! = 1 * 2 * 3 ... n)$\\
	$0 \leq k \leq n | \binom{n}{k} := \frac{k!}{k!(n-k)!}$ Binomialköffizient
\subsubsection{Lemma 39} $$1 \leq k \leq n$$ $$\binom{n+1}{k} = \binom {n}{k-1} + \binom{n}{k}$$
\begin{proof}
	$$ \binom {n}{k-1} + \binom{n}{k} = \frac{(k-1)!}{(k-1)!(n-k-1)!} + \frac{k!}{k!(n-k)!}$$
	$$ = \frac{kn! + (n -1 -k)n!}{k!(n+1-k)!} = \binom{n+1}{k}$$
\end{proof}
\subsubsection{Binomischer Lehrsatz} $\forall a,b \in \mathbb{R}$ oder $a,b, \in \mathbb{K}$ (Körper) $\forall n \in \mathbb{N}_0$\\
	$$(a+b)^n = \sum_{l=0}^n \binom{n}{l} a^{n-l}b^l$$
	$$=a^n + \binom{n}{1}a^{n-1}b^1+\binom{n}{2}a^{n-2}b^2+...+\binom{n}{n-1}ab^{n-1}+b^n$$
	\begin{proof}
	$a = 0$ klar, $a \neq 0$,$a+b)^n = a^n(1 + \frac{b}{a})^n$\\
				$\implies$ zu Zeigen: $(1 + x)^n = \sum_{l =0}{n} \binom{n}{l} x^l$
		\begin{enumerate}[a)]
			\item $n = 0$\\
				$$(1 + x)^0 = 1 = \sum_{l=0}^0 \binom{0}{l}x^l$$
			\item Induktionsannahme für n = k gilt:\\
				$$(1+x)^2 = \sum_{l=0}^k \binom{k}{l}x^l+\underbrace{\sum_{l=0}^k\binom{k}{l}x^{l+1}}_{\sum_{l=1}^{k+1}\binom{k}{l-1}x^l}$$
				$$\binom{k}{0} + \sum_{l=1}^k \binom{k}{l}x^l + \sum_{l=1}^k \binom{k}{l-1}x^l + x^{k+1} $$
				$$1+\sum_{l=1}^k\underbrace{\left(\binom{k}{l}+\binom{k}{l-1}\right)}_{=\binom{k+1}{l}}x^l + x^{l+1}$$
		\end{enumerate}
	\end{proof}
\section{Folgen und Konvergenz} $(a_1, a_2 ... a_n)$ $a_n$ Zahlen
\subsection{Definition 1} Eine Reele Folge ist eine Funktion $f: \mathbb{N} \rightarrow \mathbb{R}, n \mapsto f(n) =: a_n$
\paragraph{Schreiben:} $a_n = f(n)$, $(a_n)_{n \in \mathbb{N}}$, $(a_n)_n$
\paragraph{Bemerkung:} $(a_n)_n$ ist nicht $\{a_1, a_2, ...\}$ z.B. $a_n = 1 \implies \{a_1, a_2, ...\} = \{1\}$
\subsection{Definition 2: Konvergenz:} Sei $(a_n)_n$ eine Folge reellen Zahlen $(a_n)_n$ kovergiert gegen $L \in \mathbb{R}$\\
	Falls: $\forall \epsilon > 0 \exists k_\epsilon \in \mathbb{N}: \forall n \geq k_\epsilon: |a_n - L| < \epsilon$\\
	\paragraph{Bemerkung:} $(\forall \epsilon > 0 \exists k_\epsilon \in \mathbb{N}: \forall n \geq k_\epsilon: |a_n - L| < \epsilon)$\\
	$\Longleftrightarrow (\forall \epsilon > 0 \exists k_\epsilon \in \mathbb{N}: \forall n \geq k_\epsilon: |a_n - L| \leq \epsilon)$\\
	$\Longleftrightarrow (\forall l \in \mathbb{N} \exists k_\epsilon \in \mathbb{N}: \forall n \geq k_\epsilon: |a_n - L| < \frac{1}{l})$\\
	$\Longleftrightarrow (\forall l \in \mathbb{N} \exists k_\epsilon \in \mathbb{N}: \forall n \geq k_\epsilon: |a_n - L| \leq \frac{1}{l})$
\end{document}